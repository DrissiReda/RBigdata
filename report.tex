\documentclass{report}
\usepackage[utf8]{inputenc}
\usepackage[francais]{babel}
\usepackage[T1]{fontenc}
\usepackage{lmodern}
\usepackage{textcomp}
\usepackage{listings}
\usepackage{graphicx}
\usepackage{hyperref}
\usepackage{titlesec}
\usepackage{tcolorbox}
\usepackage{amsmath}
\usepackage{color}
\usepackage{multirow}
\setcounter{tocdepth}{5}
\setcounter{secnumdepth}{4}
\definecolor{dkgreen}{rgb}{0,0.6,0}
\definecolor{gray}{rgb}{0.5,0.5,0.5}
\definecolor{mauve}{rgb}{0.58,0,0.82}
\definecolor{gray}{rgb}{0.4,0.4,0.4}
\definecolor{darkblue}{rgb}{0.0,0.0,0.6}
\titleformat{\paragraph}
{\normalfont\normalsize\bfseries}{\theparagraph}{1em}{}
\titlespacing*{\paragraph}
{0pt}{3.25ex plus 1ex minus .2ex}{1.5ex plus .2ex}
\renewcommand{\thesection}{\Roman{section}}
\hypersetup{
    colorlinks=true,
    linkcolor=black,
    filecolor=magenta,
    urlcolor=cyan,
}
\lstnewenvironment{cc}
{
\lstset{frame=tblr,
  language=C,
  aboveskip=3mm,
  belowskip=3mm,
  showstringspaces=false,
  columns=flexible,
  basicstyle={\small\ttfamily},
  numbers=none,
  numberstyle=\tiny\color{gray},
  keywordstyle=\color{blue},
  commentstyle=\color{dkgreen},
  stringstyle=\color{mauve},
  breaklines=true,
  breakatwhitespace=true,
  tabsize=3
}}
{}

\begin{document}
\title{
  \begin{minipage}\linewidth
      \centering
      \includegraphics[width=40mm]{resources/01.png}\vskip 20pt
      R
      \vskip 5pt
      \author{
        DRISSI Mohamed Reda \\
        \texttt{reda-mohamed@isty.uvsq.fr}
      }
    \end{minipage}
}
\maketitle
\newpage
\tableofcontents
\newpage
\section{Introduction}
\section{Analyse factorielle}
Pour mieux classifier les données, et pour mieux remarquer la comparaison entre 2 variables qualitatives, nous
allons utiliser des quintiles.
\begin{verbatim}
  dfs$quintile<-5-(rank(dfs$public)-1)%/%(dim(dfs)[1]/5)
\end{verbatim}
Nous divisons les classements des films en 5 quintiles :
\begin{enumerate}
  \item Très bons
  \item Bon
  \item Moyen
  \item Pas bon
  \item Mauvais
\end{enumerate}
Pour génèrer la table de contingence avec deux variables qualitatives ce qui permet d'observer les
rapports entre les deux variables.
Test du $X^2$ : Permet de voir si les deux variables sont indépendante (retourne nul) ou pas
(retourne une valeur non nul et pas proche de 0)
\end{document}
