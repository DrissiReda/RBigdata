\documentclass{report}
\usepackage[utf8]{inputenc}
\usepackage[francais]{babel}
\usepackage[T1]{fontenc}
\usepackage[none]{hyphenat}
\usepackage{lmodern}
\usepackage{textcomp}
\usepackage{listings}
\usepackage{graphicx}
\usepackage{hyperref}
\usepackage{titlesec}
\usepackage{tcolorbox}
\usepackage{amsmath}
\usepackage{color}
\usepackage{multirow}
\setcounter{tocdepth}{5}
\setcounter{secnumdepth}{4}
\definecolor{dkgreen}{rgb}{0,0.6,0}
\definecolor{gray}{rgb}{0.5,0.5,0.5}
\definecolor{mauve}{rgb}{0.58,0,0.82}
\definecolor{gray}{rgb}{0.4,0.4,0.4}
\definecolor{darkblue}{rgb}{0.0,0.0,0.6}
\titleformat{\paragraph}
{\normalfont\normalsize\bfseries}{\theparagraph}{1em}{}
\titlespacing*{\paragraph}
{0pt}{3.25ex plus 1ex minus .2ex}{1.5ex plus .2ex}
\newcommand{\photo}[1]{\\
    \includegraphics[scale=0.5]{resources/#1.png}
\\
}
\renewcommand{\thesection}{\Roman{section}}
\hypersetup{
    colorlinks=true,
    linkcolor=black,
    filecolor=magenta,
    urlcolor=cyan,
}
\lstnewenvironment{cc}
{
\lstset{frame=tblr,
  language=C,
  aboveskip=3mm,
  belowskip=3mm,
  showstringspaces=false,
  columns=flexible,
  basicstyle={\small\ttfamily},
  numbers=none,
  numberstyle=\tiny\color{gray},
  keywordstyle=\color{blue},
  commentstyle=\color{dkgreen},
  stringstyle=\color{mauve},
  breaklines=true,
  breakatwhitespace=true,
  tabsize=3
}}
{}

\begin{document}
\title{
  \begin{minipage}\linewidth
      \centering
      \includegraphics[width=40mm]{resources/01.png}\vskip 20pt
      \Huge Big Data \\ Étude de données en R
      \vskip 5pt
      \author{
        DRISSI Mohamed Reda \\
        \texttt{reda-mohamed@isty.uvsq.fr}
      }
    \end{minipage}
}
\maketitle
\newpage
\tableofcontents
\newpage
\section{Introduction}
Pour cette étude, nous avons utilisés les données fournies qui ont été extraites depuis le site
d’Allociné. Ce dernier contient de nombreux films sortis au cinéma, ainsi que leurs critiques. De notre côté,
nous extrayons plusieurs films et leur données sorti depuis 2017 à ce jour. Les colonnes de nos données
sont classées de la manière suivante :
\begin{itemize}
  \item Le jour de sortie (jour/mois/année)
  \item La nationalité
  \item Le genre
  \item La durée du film
  \item Le nombre de fois où le film a eu $n$ étoiles (tel que $n \in [0,5]$)
  \item La note de presse (une presse par colonne)
  \item La note du public
\end{itemize}
\section{Analyse en composantes principales}
\subsection{La presse}
Le but de cette analyse est de déterminer ce qui a influencé la presse à donner une note ou une autre à un
certain film. Afin d'avoir des résultats aussi précis que possible nous allons nous limiter aux grandes presses
(i.e les presses qui ont noté le plus de films). Nous trouvons donc :
\begin{itemize}
  \item Télérama
  \item Les fiches du cinéma
  \item Le Monde
  \item Le Dauphin Libéré
  \item Le Parisien
  \item Libération
  \item La croix
  \item L'humanité
\end{itemize}
Nous utilisons donc la commande :
\begin{verbatim}
    acp1<-prcomp(~.,dfs[ ,c(14:16,18,21,24,27,29,31)],scale=TRUE)
\end{verbatim}
Les colonnes dans c() correspondent aux journaux, cette commande permet de traiter les journaux cités.
La commande prcomp permet d'effectuer une analyse en composantes principales en utilisants le SVD (singular value
decomposition).\\
Nous pouvons utiliser \texttt{summary(acp1)} pour avoir un résumé de la sortie de \texttt{prcomp}.\\

La proportion de la variance de la composante 1 représente 30\% de la variance totale. Si nous prenons les 4
premières composantes nous regroupons 70\% de la variance. Donc nous pouvons ignorer le reste, en effet
la variance cumulative sert à repérer les composantes qui sont susceptibles d'affecter nos résultats.
\photo{10}
\begin{verbatim}
  cor(dfs[rownames(acp1$x),c(14:16,18,21,24,27,29,31)])
\end{verbatim}
Cette commande donne une idée sur la corrélation entre les notes du films donnés par les journaux. Nous
remarquons qu'en moyenne les avis des journaux convergent.
Nous pouvons effectuer une rotation avec la commande \texttt{acp1\$rotation}, pour vérifier que c'est
bien la matrice de rotation nous pouvons effectuer le calcul suivant :
\begin{verbatim}
  acp1$rotation%*%diag(acp1$sdev^2)%*%t(acp1$rotation)
\end{verbatim}
Ce calcul est pratiquement celui d'une diagonalisation $P*D*P^{-1}$ puisque notre matrice de rotation est
symétrique par rapport à la diagonale, nous avons $P^{-1}=P^T$. \\
Nous trouvons ce résultat :
\photo{11}
Nous pouvons aussi examiner l'écart-type avec la commande \texttt{acp1\$sdev} (standard deviation qui
veut dire écart-type en anglais). Il est intéressant de regarder l'écart-type
pour bien apercevoir la dispersion des données
\photo{16}
Pour afficher nos mesures sur un graphique nous allons utiliser la fonction \texttt{biplot} :
\photo{12}
En effet d'après le graphique, nous remarquons que les avis convergent en moyenne, les bons films sont sur la moitié droite
(en majorité) et les mauvais films sont sur la moitié gauche.\\
L'axe rouge représente l'orientation politique des critiques, nous savons que Le Monde ou Télérama sont
des journaux de gauche, tandis que le PArisien et le Figaro sont des journaux de droite.
Puisque nous trouvons la droite vers le haut et la gauche vers le bas, nous constatons qu'il existe des bons
films pour la droite ou pour la gauche.\\
En effectuant \texttt{biplot(acp1, choices=c(1,3))} nous observons un 3ème axe qui oppose Télérama et Les fiches du cinéma.
Nous remarquons que Télérama et les Fiches du Cinéma ne sont pas d'accord sur la qualité des films, qu'importe ce que
les autres journaux pensent.
\photo{13}
\subsection{Les spectateurs}
Pour déterminer la corrélation entre les spectateurs et les notes de presses traitées précédemment nous utilisons
cette commande :
\begin{verbatim}
  cor(dfs[rownames(acp1$x),"public"],acp1$x)
\end{verbatim}
Les composantes résultantes sont :
\photo{14}
La corrélation est assez faible sur la première composante qui correspond à la qualité inhérente du film.
Donc la presse et les spectateurs ne sont pas d'accord sur la note de ce film. La composante PC2 correspondante
à l'orientation politique a la valeur maximale, ce qui est normal puisque les presse de ``droite`` seront
du même avis que les spectateurs de droite.
Pour afficher une 3ème variable sur notre schéma de critiques en focntion des films (critique des spectateurs)
Nous allons utiliser cette fonction :
\begin{verbatim}
  fleche<-function(nom,axes=c(1,2)){
    coord<-cov(dfs[rownames(acp1$x),nom],acp1$x)
    arrows(0,0,coord[axes[1]]*acp1$sdev[1],coord[axes[2]]*acp1$sdev[2]*5,
      col="blue")
    sum (coord^2)
}
\end{verbatim}
Le graphique résultant est :
\photo{15}
Nous trouvons donc que le public se range du côté des journaux de droite, cela nous amène à conclure
que le public est plus de ``droite``.
Avant d'effectuer notre analyse des composantes principales avec \texttt{PCA} nous allons compléter notre
jeu de données à l'aide de cette commande :
\begin{verbatim}
  rescomp<-imputeFAMD(dfs[,c(5,14:31,95:97)],ncp=2)
\end{verbatim}
Puis nous pouvons utiliser la fonction \texttt{PCA}
\begin{verbatim}
  acp2<-PCA(rescomp$completeObs, quanti.sup=c(20,21), quali.sup=c(1,22))
\end{verbatim}
Le résultat est 2 diagrammes :
\photo{17}
Nous remarquons que quelques pleins n'appartiennent pas au nuage constituant la plupart des films.
Ces points qui échappent à la règle sont les seuls de leur catégorie, donc nous ne pouvons pas
les comparer aux autres catégories.
\photo{18}
Dans ce second diagramme, les flèches vont dans des directions similaires, ce qui veut dire que
les avis convergent globalement
Nous allons enlever les films excentrés du premier graphes, soit Un beau Soleil intérieur et Molly
\photo{19}
Après avoir récupéré les lignes Molly et Un beau Soleil intérieur, nous exécutons la commande :
\begin{verbatim}
  comp2<-PCA(rescomp$completeObs, ind.sup=c(742,54), quanti.sup=c(20,21), quali.sup=c(1,22))
\end{verbatim}
Individus :
\photo{191}
Variables : 
\photo{192}
Le deuxième graphe change légèrement au niveau du \textit{varpub}
\newpage
\section{Analyse factorielle des composantes}
\subsection{Analyse des nationalités}
Pour mieux classifier les données, et pour mieux remarquer la comparaison entre 2 variables qualitatives, nous
allons utiliser des quintiles.

\begin{verbatim}
  dfs$quintile<-5-(rank(dfs$public,useNA="no")-1)%/%(dim(dfs)[1]/5)
\end{verbatim}
Nous divisons les classements des films en 5 quintiles :
\begin{enumerate}
  \item Mauvais
  \item Pas bon
  \item Moyen
  \item Bon
  \item Très bons
\end{enumerate}
Le résultat de \texttt{dfs\$quintile} donne :
\photo{30}
Pour génèrer la table de contingence avec deux variables qualitatives ce qui permet d'observer les
rapports entre les deux variables, nous utilisons cette commande.
\begin{verbatim}
  tab<-table(dfs$nat2,dfs$quintile)
\end{verbatim}
Nous trouvons :
\photo{31}
De première vue nous avons déjà une certaine idée de la qualité des films selon leurs nationalités.
Par exemple les films américains sont peu présent dans le premier rang (mauvais films), et les français
ont plein de films dans le plus de films dans les 2 premiers quintiles. Cette analyse va nous permettre
d'essayer de comprendre la raison.
Test du $\chi^2$ : Permet de voir si les deux variables sont indépendantes (p-value <0.05) ou pas
(retourne une valeur >0.05). En effet nos 2 variables ne sont pas indépendantes puisque
la commande \texttt{chisq.test(tab)} retourne un $p-value=7.065\times10^{-9}$
comme le montre la capture d'écran :
\photo{32}
Pour effectuer notre analyse factorielle des correspondances nous utilisons la commande \texttt{afc<-CA(tab)}
\photo{33}
Chaque nationalité forme un groupe de films distincts. Nous remarquons comme précédemment vu sur le
tableau de contingence que les américans ont plus de bons films, tandis que les français ont plus
de mauvais films, les films de nationalité différentes des 2 précédentes sont moyens.
Nous savons que notre tableau est cohérent car la somme des 2 dimensions est de $100\%$ .

Pour davantage de clarité nous allons pondérer les nationalités en fonction des quintiles,
Cela va nous donner les coordonnées des profils lignes à l’intérieur des profils colonnes
Pour afficher notre résultat :
\begin{verbatim}
  res<-tab%*%afc$col$coord/apply(tab,1,sum)
  (afc$col$coord[ ,c(1,2)])
  points(res,col="blue"). #ajout du résultat de la première commande dans le plot
\end{verbatim}
Le graphique résultant :
\photo{34}
Grâce à cette pondération, notre graphique offre davantage d'information sur la proximité entre
les nationalités et leurs quintiles ``chauds``.
\begin{itemize}
  \item Américains : entre les quintiles 4 et 5
  \item Français : entre les quintiles 1 et 2
  \item Autres : entre les quintiles 2 et 4
\end{itemize}
Nous observons les ellipses correspondantes aux intervalles de confiance. Ce sont les ellipsoïdes
de confiance pour chacun des points.
Nous utilisons la commande \texttt{ellipseCA()} :
\photo{36}
Nous remarquons que la modalité des films américains est très différente des autres, D'une autre
part les modalités des films français et de nationalité autre ont des ellipsoïdes qui s'interceptent,
cependant nous ne pouvons pas supposer qu'ils sont confondus, ni supposer qu'ils sont séparés
statistiquement. Les films du 1er quintiles sont aussi suffisament loins des ellipsoïdes des nationalités
pour supposer qu'ils sont d'une nationalité différente du reste.
\subparagraph{Conclusion} \begin{verbatim}\end{verbatim}
Il y a pleins de bons films américains et pleins de mauvais films français, ce qui est normal,
puisque les salles françaises ne vont jamais investir dans de mauvais films américains, ce qui
justifie l'absence de ces derniers dans notre échantillon. Par contre les films français, bons ou
mauvais seront forcèment visualisés en france (local) puisque comme les films américains, ils n'auront
aucun succès et aucune attirance pour des salles internationales.
\subsection{Analyse des genres}
De même nous allons observer la table de contingence entre les quintiles et les genre de films,
nous commençons par créer cette table :
\begin{verbatim}
  tab2<-table(dfs$genre,dfs$quintile)
\end{verbatim}
Résultat :
\photo{35}
Nous remarquons que quelques genres ont très peu de films donc nous ne pouvons pas considérer
ces films comme représentatifs de leurs genres, l'échantillon est trop petit et la probabilité que
nos résultats soient faux est trop grande.
Nous allons donc enlever ces genres avec la commande
\begin{verbatim}
  tab2mod <-tab2[-c(7,8,13,17,22),]
\end{verbatim}
Le test du $\chi^2$ génère un warning : Chi-squared approximation may be incorrect.\\
Donc nous allons essayer avec l'option : \texttt{simulate.p.value = TRUE}
Nous trouvons :
\photo{37}
La p-value est de  $0.0004998$ donc comme avec les nationalités, nous trouvons que nos 2 variables
ne sont pas indépendantes.
Nous effectuons l'analyse factorielle avec la commande \texttt{CA()} et nous trouvons :
\photo{38}
Comme avec les nationalités nous allons essayer de trouver les genres appréciés et ceux qui ne le
sont pas.
\begin{verbatim}
  res2<-tab2mod%*%afc2$col$coord/apply(tab2mod,1,sum)
  (afc2$col$coord[,c(1,2)])
  points(res2,col="green")
\end{verbatim}
Le résultat :
\photo{39}
Nous remarquons que les films de guerre recouvrent complètement le quintile 4 donc ces films sont tous
bons. Les filmes romantiques d'une autre part ont des qualités mixés puisqu'ils sont soit trop bons
soit trop mauvais (quintile 1 et 5). Les films thrillers et science fiction  sont très présents dans
les quintiles supérieurs ce qui veut dire qu'ils sont très appréciés par le public.

\end{document}
